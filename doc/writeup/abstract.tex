\documentclass[a4paper]{article}
%\usepackage{simplemargins}

%\usepackage[square]{natbib}
\usepackage{amsmath}
\usepackage{amsfonts}
\usepackage{amssymb}
\usepackage{graphicx}

\begin{document}
\pagenumbering{gobble}

\Large
 \begin{center}
Generating Synthetic Electric Distribution Networks for Montgomery County of Virginia\\ 

\hspace{10pt}

% Author names and affiliations
\large
Rounak Meyur$^1$, Anil Vullikanti$^2$, Henning Mortveit$^2$, Samarth Swarup$^2$ and Virgilio Centeno$^1$ \\

\hspace{10pt}

\small  
$^1$ ECE Department, Virginia Tech\\
rounakm8@vt.edu\\
$^2$ Network Simulation Science and Advanced Computing  Division (NSSAC), University of Virginia

\end{center}

\hspace{10pt}

\normalsize
The electric power system is considered as a critical infrastructure providing crucial support to the economy and society of a country. Therefore, any data relating to the actual power system is considered highly sensitive information and is restricted through a non-disclosure agreement. This necessitates the requirement to generate fictitious power system networks which mimics the features of an actual power grid, barring the sensitive information. Most of the prior works in the topic of generating synthetic power system networks has been concentrated at the transmission level. In this work, the authors propose a methodology to generate synthetic power distribution networks using open source information such as location of high voltage substations, census information and NAVTEQ transportation network data. The homes and other activity locations are first mapped to the NAVTEQ road network nodes based on the geometric distance between them. In this case, the NAVTEQ road network edges are considered as proxy for the branches of the primary distribution network with the assumption that pole-top transformers are located at the road network nodes. The hourly load demand profiles are synthetically generated for each home/activity location and the ratings of pole-top transformers are determined based on the diversity factor of the load to be served. The proposed methodology generates a primary distribution system connecting the substations to the pole top transformers by treating it similar to an electric distribution system planning problem. This is formulated as a MILP problem aiming to optimize the overall length of conductors in the network. The secondary distribution network connecting the transformers to the loads is constructed based on a proposed heuristic to maintain equal distribution of load across three phases. The methodology would be implemented for generating the synthetic distribution network of Montgomery county of Virginia. The results would be validated based on the voltage profile at different locations in the network when operating at various loading conditions. Such a method would facilitate researchers to build realistic distribution networks for analysis and avoid the small sized IEEE test case systems.

\end{document}
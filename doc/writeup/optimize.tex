\section{Problem Formulation}
Using the mapping proposed in the previous section, we can consider nodes in the road network to be nodes in the distribution grid. Let $\mathsf{\Omega_b}$ denote the set of all nodes in the road network along with the substation buses. Let $\mathsf{\Omega_{bs}}\subset\mathsf{\Omega_b}$ denote the set of substation buses. Furthermore, the mapping $p^{-1}$ between road nodes and activity locations can be used to evaluate the load demand at each node. Let $P_{Di}$ and $P_{Si}$ denote the load and power supplied at node $i$ for $i\in\mathsf{\Omega_b}$. It is obvious that $P_{Si}=0$ for $i\notin\mathsf{\Omega_{bs}}$. Also, let $\mathsf{\Omega_{bp}}$ denote the set of transfer nodes where $P_{Di}=0$. These are the nodes in the road network which are not mapped to any activity locations.

The next step is to formulate an algorithm to construct the connected graph between these nodes maintaining a radial structure. Let $\mathsf{\Omega_l}$ denote the set of all possible edges between the nodes. The problem of generating a synthetic distribution network can be considered as a variation of the distribution system planning (DSP). The problem is formulated as follows
\begin{equation}
\min\quad f=\sum_{(ij)\in\mathsf{\Omega_l}}{c_{l,ij}\delta_{ij}l_{ij}}+\sum_{(ij)\in\mathsf{\Omega_l}}{g_{ij}\delta_{ij}(V_i^2+V_j^2-2V_iV_j\cos\theta_{ij})}+\sum_{i\in\mathsf{\Omega_{bs}}}{c_{v,i}(P_{Si}^2+Q_{Si}^2)}\label{eq:obj}
\end{equation}
The three terms in the objective function are respectively the cost of installing a branch between nodes $i$ and $j$, the power loss in the branch $i,j$ and the cost of operation of substation at node $i$. The cost of operating a substation is dependent on its power rating. 

Now, the constraints are formulated for the problem. First, node power balance equations are listed
\begin{equation}
\begin{aligned}
P_{Si}-P_{Di}-\sum_{j\in\mathsf{\Omega_{b}}}{\delta_{ij}P_{ij}}=0\quad\forall i\in\mathsf{\Omega_{b}}\\
Q_{Si}-Q_{Di}-\sum_{j\in\mathsf{\Omega_{b}}}{\delta_{ij}Q_{ij}}=0\quad\forall i\in\mathsf{\Omega_{b}}
\end{aligned}
\label{eq:node-balance}
\end{equation}
Now the real and reactive power flow in a branch is given by
\begin{equation}
\begin{aligned}
	P_{ij}&=V_i^2g_{ij}-V_iV_j(g_{ij}\cos\theta_{ij}+b_{ij}\sin\theta_{ij})\quad\forall (ij)\in\mathsf{\Omega_l}\\
	Q_{ij}&=-V_i^2b_{ij}-V_iV_j(g_{ij}\sin\theta_{ij}-b_{ij}\cos\theta_{ij})\quad\forall (ij)\in\mathsf{\Omega_l}
\end{aligned}
\label{eq:power-flow}
\end{equation}
The voltage, substation rating and line flow constraints are listed now
\begin{equation}
\underline{V}\leq V_i\leq\overline{V}\quad i\in\mathsf{\Omega_b} \label{eq:volt-constraint}
\end{equation}
\begin{equation}
0\leq P_{Si}^2+Q_{Si}^2\leq \overline{S}^2\quad \forall i\in\mathsf{\Omega_{bs}} \label{eq:substation-rating}
\end{equation}
\begin{equation}
0\leq P_{ij}^2+Q_{ij}^2\leq \overline{S_f}^2\quad \forall (ij)\in\mathsf{\Omega_{l}} \label{eq:line-flow}
\end{equation}
Finally, the decision variable $\delta_{ij}$ is a binary variable.
\begin{equation}
\delta_{ij}\in\{0,1\}\quad\forall (ij)\in\mathsf{\Omega_{l}}\label{eq:decision}
\end{equation}
The following equations are used to formulate the radiality constraint for the distribution network. In this case, we consider the transfer nodes which have neither generation nor loads connected to it. We define binary variable $\epsilon_j$ for each transfer node. The transfer node is used to connect a load node to other load nodes. It is also not a terminal node. The value $\epsilon_j$ is $1$ if the transfer node is used, otherwise it is set to $0$.
\begin{equation}
\begin{aligned}
&\sum_{(i,j)\in\mathsf{\Omega_l}}\delta_{ij}=n_b-n_{bs}-\sum_{i\in\mathsf{\Omega_{bp}}}(1-\epsilon_j)&\\
&\delta_{ij}\leq\epsilon_j&\quad \forall (ij)\in\mathsf{\Omega_{l}},\forall j\in\mathsf{\Omega_{bp}}\\
&\delta_{ji}\leq\epsilon_j&\quad \forall (ji)\in\mathsf{\Omega_{l}},\forall j\in\mathsf{\Omega_{bp}}\\
&\sum_{(ij)\in\mathsf{\Omega_l}}\delta_{ij}+\sum_{(ji)\in\mathsf{\Omega_l}}\delta_{ji}\geq 2\epsilon_j&\quad \forall j\in\mathsf{\Omega_{bp}}\\
&\epsilon_{j}\in\{0,1\}&\quad \forall j\in\mathsf{\Omega_{bp}}
\end{aligned}
\label{eq:radiality}
\end{equation}
\newpage
\section{Simplified Problem Formulation}
We can simplify the planning problem to avoid the reactive power terms by assuming a power factor of $0.8$ throughout the system. This reduces the problem to
\begin{equation}
\begin{aligned}
&\min\quad\  f=\sum_{(ij)\in\mathsf{\Omega_l}}{c_{l,ij}\delta_{ij}l_{ij}}+\sum_{(ij)\in\mathsf{\Omega_l}}{g_{ij}\delta_{ij}(V_i^2+V_j^2-2V_iV_j\cos\theta_{ij})}+\sum_{i\in\mathsf{\Omega_{bs}}}{c_{v,i}P_{Si}}\label{eq:simple-1}\\
&s.to.\quad\ P_{Si}-P_{Di}-\sum_{j\in\mathsf{\Omega_{b}}}{\delta_{ij}P_{ij}}=0\quad\forall i\in\mathsf{\Omega_{b}}\\
&\quad\quad\quad P_{ij}=V_i^2g_{ij}-V_iV_j(g_{ij}\cos\theta_{ij}+b_{ij}\sin\theta_{ij})\quad \forall (ij)\in\mathsf{\Omega_l}\\
&\quad\quad\quad \underline{V}\leq V_i\leq\overline{V}\quad i\in\mathsf{\Omega_b}\\
&\quad\quad\quad 0\leq P_{Si}\leq \overline{P}\quad \forall i\in\mathsf{\Omega_{bs}}\\
&\quad\quad\quad 0\leq P_{ij}\leq \overline{P_f}\quad \forall (ij)\in\mathsf{\Omega_{l}}\\
&\quad\quad\quad\sum_{(i,j)\in\mathsf{\Omega_l}}\delta_{ij}=n_b-n_{bs}-\sum_{i\in\mathsf{\Omega_{bp}}}(1-\epsilon_j)\\
&\quad\quad\quad\delta_{ij}\leq\epsilon_j\quad \forall (ij)\in\mathsf{\Omega_{l}},\forall j\in\mathsf{\Omega_{bp}}\\
&\quad\quad\quad\delta_{ji}\leq\epsilon_j\quad \forall (ji)\in\mathsf{\Omega_{l}},\forall j\in\mathsf{\Omega_{bp}}\\
&\quad\quad\quad\sum_{(ij)\in\mathsf{\Omega_l}}\delta_{ij}+\sum_{(ji)\in\mathsf{\Omega_l}}\delta_{ji}\geq 2\epsilon_j\quad \forall j\in\mathsf{\Omega_{bp}}\\
&\quad\quad\quad\epsilon_{j}\in\{0,1\}\quad \forall j\in\mathsf{\Omega_{bp}}\\
&\quad\quad\quad\delta_{ij}\in\{0,1\}\quad\forall (ij)\in\mathsf{\Omega_{l}}
\end{aligned}
\end{equation}
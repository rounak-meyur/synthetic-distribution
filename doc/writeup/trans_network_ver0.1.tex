\documentclass[10pt,conference]{IEEEtran}
\IEEEoverridecommandlockouts
\usepackage{cite}
\usepackage{amsmath,amssymb,amsfonts,amsthm}
\DeclareMathOperator*{\argmax}{arg\,max}
\DeclareMathOperator*{\argmin}{arg\,min}
\usepackage{algorithmic}
\usepackage{graphicx,subcaption}
\usepackage{textcomp}
\usepackage{xcolor}
\newtheorem{definition}{\textbf{Definition}}[section]
\def\BibTeX{{\rm B\kern-.05em{\sc i\kern-.025em b}\kern-.08em
		T\kern-.1667em\lower.7ex\hbox{E}\kern-.125emX}}

\usepackage{hyperref}
\hypersetup{colorlinks=true,linkcolor=blue,linkcolor=blue,citecolor=blue,urlcolor=blue}

%%
\graphicspath{{./}{./figs/}}

%%\usepackage{citesort}
%\usepackage{sidecap}
%\usepackage{times} % assumes new font selection scheme installed
%\usepackage[small]{caption}




\renewcommand{\thesection}{\Roman{section}}
\renewcommand{\thesubsection}{{\bfseries\Alph{subsection}}}




%%
\begin{document}
	\title{Creating synthetic transmission networks}
	
	\author
	{
		\IEEEauthorblockN
		{
			Rounak Meyur\IEEEauthorrefmark{1},Shivani Garg\IEEEauthorrefmark{2},Rachel Szabo\IEEEauthorrefmark{2},Srijan Sengupta\IEEEauthorrefmark{2} and Virgilio Centeno\IEEEauthorrefmark{1}
		}
		\IEEEauthorblockA
		{
			\IEEEauthorrefmark{1}\textit{ECE Department, Virginia Tech}, Blacksburg, USA\\
			\IEEEauthorrefmark{2}\textit{Department of Statistics, Virginia Tech}, Blacksburg, USA
		}
	}
	\maketitle
	
	\begin{abstract}
		This paper
	\end{abstract}

	\begin{IEEEkeywords}
		synthetic power networks, clustering, data analytics
	\end{IEEEkeywords}
	
	\section{Creation of synthetic power grid}\label{sec:synthetic}
	A power system network constitutes a collection of buses which are interconnected by a set of transmission lines and transformers. This can be represented by a connected graph $\mathsf{G}(\mathsf{V},\mathsf{E})$ where $\mathsf{V}$ denotes the set of nodes and $\mathsf{E}$ is the set of edges connecting the nodes. In this paper, a two-step method is proposed to generate realistic synthetic power system topologies from publicly available information such that they bear resemblance to real power grids. In the first step of the proposed methodology, substations would be placed at particular geographical locations based on data sets published by U.S. Energy Information Administration (EIA). These substations may consist of generators, loads or both. Statistical analysis on typical power system networks such as the Eastern Interconnect (EI) shows that nodes predominantly serve two purposes~\cite{overbye_102}: 
	\begin{itemize}
		\item supply power into the grid (generator nodes), or 
		\item provide electricity to the associated downstream distribution network in order to meet the load demand of customers (load nodes).
	\end{itemize}
	Therefore, most nodes in the synthetic power system model would either be load or generator nodes. However, a substation might have multiple buses with neither load nor generation, for example buses which are connected to transformers. The load and generator nodes are determined based on other open source information from U.S. Census data and U.S. EIA data sets which are described below in detail.
	
	In the next step, the nodes are to be connected through transmission lines and transformers. For this purpose, the voltages of all nodes in the network are determined. Thereafter, nodes with similar voltage levels are connected by transmission lines, while transformers connect buses with different voltage levels.
	\subsection{Loads}\label{ssec:load}

	
	\bibliographystyle{IEEEtran}
	\bibliography{references}
\end{document}
\section{Preliminaries}\label{sec:prelim}
In this work, we try to generate the distribution network for a given region (county/town/city) from different open source publicly available information. These data pertain to following sources.
\begin{itemize}
	\item Transportation network data published by NAVTEQ~\cite{navteq}.
	\item Geographical location of high voltage (HV) and extra-high voltage (EHV) substations from data sets published by U.S. Energy Information Administration (EIA)~\cite{eia_substations}.
	\item Residential electric power demand information developed in the models by~\cite{swapna_2018}.
\end{itemize}
The distribution system is synthesized in a way such that it resembles a typical radial distribution feeder network. The goal of this work is to generate the primary and secondary distribution network to connect the substations to all building locations. We want a mapping between the substations and residential locations. This can be achieved by mapping the substations to nodes of the transportation (road) network and assigning a map between residential buildings and road network links. The road network can be used as a proxy for the primary distribution network. The secondary distribution network can be generated from the second map.

The NAVTEQ transportation network data is available in the form of an edge-list or a list of links. Each link $e$ has an associated integer level $l_e$ from the set $\{1,2,3,4,5\}$ that describes the link type (e.g., a level-1 ($l_e=1$) link could correspond to an Interstate road segment, while a level-5 ($l_e=5$) link could correspond to a residential driveway). All links with level $l_e\leq2$ are removed from the dataset. These links are dropped since	components of the distribution network (e.g., homes) are typically not located along these links. Furthermore, connected components in the road network graph of considerable size (number of nodes greater than 10) are only considered. This is done to remove small connected components along which a radial primary distribution feeder network cannot exist.

The synthetic population data and the substation information is available for the entire United States and stored in a database together with the NAVTEQ data. The synthetic distribution network is generated for a particular county. For this purpose, the required data set is selected from the master database based on the zip code information. The available data set is therefore listed as below.
\begin{enumerate}
	\item[(i)] The road network represented in the form of a graph $\mathsf{R}=(\mathsf{V},\mathsf{L})$, where $\mathsf{V}$ and $\mathsf{L}$ are respectively the sets of nodes and links of the network. Each node in the graph has an associated spatial embedding in form of longitude and latitude. 
	\item[(ii)] The set of substations $\mathsf{S}=\{s_1,s_2,\cdots,s_M\}$, where the county consists of $M$ substations and their respective geographical location data.
	\item[(iii)] The set of residential building locations $\mathsf{H}=\{h_1,h_2,\cdots,h_N\}$, where the county consists of $N$ home locations. Each building is associated with longitude-latitude information.
\end{enumerate}

